\documentclass[11pt,a4paper]{colorart}

\usepackage{graphicx}
\usepackage{amsmath}
\usepackage{bm}
\usepackage{latexsym}
\usepackage{amsfonts,amssymb}
\usepackage{psfrag}
\usepackage{ulem}
\usepackage{textcomp}
\usepackage{xcolor}
\usepackage{physics}
\usepackage{amsthm}
\usepackage{mathrsfs}
\usepackage{tikz-cd}
\usepackage{import, pdfpages, transparent}
\usepackage{caption, subcaption}
\usepackage[section]{placeins}
\usepackage{hyperref}

\def\nn{\nonumber} 
\def\pa{{\partial}}
\def\f{\frac}
\def\l{\left}
\def\r{\right}
\def\d{{\delta}}
\def\es{\emptyset}
\def\R{\mathbb{R}}
\def\C{\mathbb{C}}
\def\N{\mathbb{N}}
\def\Z{\mathbb{Z}}
\def\Q{\mathbb{Q}}
\def\F{\mathbb{F}}
\def\D{\mathfrak{D}}
\def\e{\epsilon}
\def\a{\alpha}
\def\b{\beta}
\def\g{\gamma}
\def\p{\phi}
\def\Ker{\text{Ker}}
\def\Con{\text{Con}}
\def\L{\mathcal{L}}
\def\ra{\rightarrow}

\title{\huge Analysis 2}
\author{Incalculas}
\date{Updated on: \today}

\begin{document}

\maketitle
\tableofcontents
\newpage

\section{Differentiation}

\subsection{Definition}

\begin{definition}[Differentiation]
	Let $J$ be an interval in $\R$ with $c\in J$ and let $f:J\ra \R$ be a function from $J$ to $\R$.
	The function is said to be differentiable at $c$ if 
	
	\[ \lim_{ x \to c } \frac{ f(x) - f(c) }{ x - c } \]
	
	exists.

\end{definition}

\begin{notation}[Differentiation]
	If $f$ is differentiable at $c$, we define 
	
	\[ \lim_{ x \to c } \frac{ f(x) - f(x) }{ x - c } 
	   = \frac{ df }{ dx } \biggr|_{ x=c } = f'(c) \]
\end{notation}

Let's now discuss about interpretation of differentiation\\

Let $J$ be an interval in $R$ with $c\in J$ and let $f:J\ra \R$ be a function.\\

Aim is to find an approximation for $f$ near 0, $f(x) \approx a + bx$ for some $a,b \in \R$ near 0.

\[ \lim_{x \to c} \l[ f(x) - (a+bx) \r] = 0 \]

\[ E(x) = f(x) - (a+bx) \]

Find $a,b \in \R$ such that,

\[ \lim_{x \to c} E(x) = 0 \]

Then, $a=f(c)$ and $b=f'(c)$, this tells us  that we know first order approximation from differentiation. 

\begin{definition}[Alternate definition for differentiation]
	Let $J$ be an interval in $\R$ with $c \in J$ and let $f:J \ra \R$, we say $f$ is differentiable if and only if 
	\[ \lim_{h \to 0} \frac{ f(c+h) - f(c) }{ h } \]
	exists.
\end{definition}

\begin{remark}
	Take $J = [a,b]$ then
	\begin{enumerate}
		\item $c=a$, we only check for one sided limit (right hand limit)
		\item $a<c<b$, we check both left hand and right hand limits
		\item $c=b$, we only check for one sided limit (left hand limit)
	\end{enumerate}
\end{remark}

\begin{definition}[Differentiable on an interval]
	Let $J$ be an interval on $R$ and $f:J \ra \R$ be real function on $J$, we say $f$ is differentiable on $J$ if $f$ is differentiable for every point on $J$.
\end{definition}

\begin{example}
	$f(x) = x$, 
	\[ f'(x) = \lim_{h \to 0} \frac{ (x+h) - (x) }{ h} 
	   = \lim_{h \to 0} 1 = 1 \implies f'(x) = 1 \]
\end{example}

\subsection{Equivalent condition for differentiation}

\begin{theorem}[Equivalent condition for differentiation]
	Let $J$ be an interval in $\R$ with $c \in J$ and let $f:J \ra \R$ then $f$ is differentiable at $c$ if and only if $\exists f_1:J \ra \R$ such that
	\begin{enumerate}
		\item $f(x) = f(c) + f_1(x)  (x-c) \forall x \in J$.
		\item $f_1$ is continous at $c$.
	\end{enumerate}
	and $f_1(c) = f'(c)$
\end{theorem}

\begin{proof}
	Let $f$ be differentiable at $c$ then
	
	\[ \lim_{ x \to c } \frac{ f(x) - f(c) }{ x - c} = f'(c) \]
	
	Now define, 

	\[ f_1(x) = \l\{ 
		\begin{array}{ll}
			\frac{ f(x) - f(c) }{ x -c } & x \neq c\\ 
	   		f'(c) & x=c 
		\end{array} 
		     \r. \]
	This now proves i) and ii).\\

	Assume that $f_1$ exists and satisfies i) and ii) then,
	
	\[ f_1(x) = \frac{ f(x) - f(c) }{ x-c }, x \neq c \]
	
	and since $f_1(x)$ is continous at $c$ 
	
	\[ \lim_{x \to c} f_1(x) \]
	
	exists and therefore,
	
	\[ f_1(x) = \lim_{ x \to c } f_1(x) = \lim_{x \to c} \frac{ f(x) - f(c) }{ x-c } \]

	This proves that $f$ is differentiable at $c$ and $f'(c) = f_1(c)$.
\end{proof}

\begin{corollary}
	Let $f:J\ra\R$ be function from interval $J\in\R$ to $\R$ be differential at $c\in J$ then $f$ is continous at $c$.
\end{corollary}

\begin{proof}
	From Theorem 1.7,
	\[ f(x) = f(c) + f_1(x)  (x-c) \]
	$f_1$ is contious at $c$ and hence $f$ is differentiable at $c$
\end{proof}

\begin{example}
	Define $f: \R \ra \R, x \ra x^n$ for some fixed $n \in \N$, for $c \in \R$, $f'(c) = nc^{n-1}$.
\end{example}

\begin{proof}
	
	\[ f(x) - f(c) = x^n - c^n \]
	
	\[ \implies f(x) - f(c) = ( x-c ) ( x^{n-1} + c x^{n-2} + c^2 x^{n-3} + \dots + c^{n-1} ) \]

	Let $f_1(x) = x^{n-1} + c x^{n-2} + c^2 x^{n-3} + \dots + c^{n-1}$, given it's a polynomial, we know $f_1(x)$ is continous in $\R$.

	\[ f_1(c) = nc^{n-1} \]

	Therefore $f$ is differentiable in $\R$ and $f'(x) = nx^{n-1}$.

\end{proof}

\begin{example}
	Define $f: \R \ra \R, x \ra e^x$, for $c \in \R$, $f'(c) = e^c$.
\end{example}
	
\begin{proof}
	
	\[ f(x) - f(c) = e^x - e^c \]

	\[ \implies f(x) - f(c) = e^c \l( e^x - 1 \r) \]

	Since, 
	
	\[ e^x -1 = \sum_{n=1}^{\infty} \frac{ x^n }{ n! } = x \sum_{n=1}^{\infty} \frac{ x^{n-1} }{ n! } \]

	Define,

	\[ f_1(x) = e^c \sum_{n=1}^{\infty} \frac{ (x-c)^{n-1} }{ n! } \]

	Assume that $f_1$ is continous at $c$,

	\[ f_1(c) = e^c \]

	This proves that $f$ is differentiable at $c$ and $f'(c) = e^c$.\\

	(Note: this proof uses power series of exponential function which uses differentiation, this proof is just a circular argument and we will get back to an actual proof at the end of the course)

\end{proof}

\begin{example}
	Define $f: \R \ra \R, x \ra |x|$, for $c \in \R$, $f'(c) = 1$ if $c>0$, $f'(c) = -1$ if $x<0$ and $f'(c)$ does not exist for $c=0$.
\end{example}

\begin{proof}
	Both $c<0$ and $c>0$ cases are trivial. Now assume $f(x)$ is differentiable at 0 then the following limit exists.
	
	\[ \lim_{ h \to 0 } \frac{ |0+h| -|0| }{ h } = |h|/h \]
	
	Clearly, left hand limit is $-1$ and right hand limit is $1$ and therefore limit does not exist.

\end{proof}

\begin{theorem}[Algebraic properties]
	Let $J$ be an interval in $\R$ and let $f:J \ra \R$ and $g:J \ra \R$ be 2 functions differentaible at $c\in J$.
	\begin{enumerate}
		\item $(f+g)$ definied as 
			\[ (f+g)(x) = f(x) + g(x) \] 
			Then,
			\[ (f+g)'(c) = f'(x) + g'(x) \]
		\item for $\a \in \R$, $(\a f)$ definied as 
			\[ (\a f)(x)=\a f(x) \]
			Then,
			\[ (\a f)'(c)=\a f'(x) \]
		\item $(fg)$ definied as 
			\[ (fg)(x)=f(x)g(x)\] 
			Then,
			\[ (fg)'(c)=f'(c)g(c)+f(c)g'(c) \]
		\item Assume $f(c)\neq 0$, $(\frac{1}{f})$ definied as 
			\[ \l(\frac{1}{f}\r)(x) = \frac{1}{f(x)} \]
			Then,
			\[ \l(\frac{1}{f}\r)'(c) = -\frac{f'(c)}{(f(c))^2} \]
	\end{enumerate}
\end{theorem}

\begin{proof}
	For \textit{i)} :\\
	From theorem 1.7 we have that there exists continous functions $f_1:J\ra\R$ and $g_1:J\ra\R$ such that,
	\[ f(x) = f(c) + f_1(x)  (x-c) \]
	\[ g(x) = g(c) + g_1(x)  (x-c) \]
	From these equations we have,
	\[ f(x) + g(x) = f(c) + g(c) + f_1(x)  (x-c) + g_1(x) (x-c) \]
	\[ \implies (f+g)(x) = (f+g)(c) +  ( f_1(x)  + g_1(x) ) (x-c) \]
	\[ \implies (f+g)(x) = (f+g)(c) +  ((f_1+g_1)(x)) (x-c) \]
	Now applying theorem 1.7, we have that
	\[ (f+g)'(c) = (f_1+g_1)(c) = f_1(c) + g_1(c) = f'(c) + g'(c) \]
	\[ \implies (f+g)'(c) = f'(c) + g'(c) \]
	We shall apply the same structure for the proofs of other algebraic properties,\\
	For \textit{ii)} :\\
	From theorem 1.7 we have,
	\[ f(x) = f(c) + f_1(x)  (x-c) \]
	From this equation we have,
	\[ \a f(x) = \a f(c) + \a f_1(x)  (x-c) \]
	\[ \implies (\a f)(x) = (\a f)(c) + ( (\a f_1)(x) ) (x-c) \]
	Now applying theorem 1.7, we have that
	\[ (\a f)'(c) = (\a f_1)(c) = \a f_1(c) = \a f'(c) \]
	\[ \implies (\a f)'(c) = \a f'(c) \]

	For \textit{iii)} :\\
	From theorem 1.7 we have,
	\[ f(x) = f(c) + f_1(x)  (x-c) \]
	\[ g(x) = g(c) + g_1(x)  (x-c) \]
	From these equations we have,
	\[ f(x)g(x) = f(c)g(c) + (x-c) \l[ f(c)g_1(x) + f_1(x)g(c) + f_1(x)g_1(x)(x-c) \r] \]
	Now applying theorem 1.7,
	\[ (fg)'(c) = f(c)g_1(c) + f_1(c)g(c) \]
	\[ \implies (fg)'(c) = f(c)g'(c) + f'(c)g(c) \]

	For \textit{iv)} :\\
	From theorem 1.7 we have,
	\[ f(x) = f(c) + f_1(x)  (x-c) \]
	\[ \implies \frac{1}{f(x)} - \frac{1}{f(c)} = \frac{1}{f(c)+f_1(x)(x-c)} -\frac{1}{f(c)} \]
	\[ \implies \frac{1}{f(x)} - \frac{1}{f(c)} =  \frac{f(c)-f(c)-f_1(x)(x-c)}{(f(c))(f(c)+f_1(x)(x-c))}\]
	\[ \frac{1}{f(x)} = \frac{1}{f(c)} -  \frac{-f_1(x)(x-c)}{(f(c))(f(c)+f_1(x)(x-c))}\]
	\[ \implies \l(\frac{1}{f}\r)(x) = \l(\frac{1}{f}\r)(c) -  \frac{-f_1(x)(x-c)}{(f(c))(f(c)+f_1(x)(x-c))}\]
	\[ \implies \l(\frac{1}{f}\r)(x) = \l(\frac{1}{f}\r)(c) -  (x-c)\frac{-f_1(x)}{(f(c))(f(c)+f_1(x)(x-c))}\]
	Now applying theorem 1.7,
	\[ \l(\frac{1}{f}\r)'(c) = -  \frac{f_1(c)}{(f(c))(f(c)+f_1(c)(c-c))} \]
	\[ \implies \l(\frac{1}{f}\r)'(c) = -  \frac{f_1(c)}{(f(c))^2} \]
	\[ \implies \l(\frac{1}{f}\r)'(c) = -  \frac{f'(c)}{(f(c))^2} \]
\end{proof}

\subsection{Differentiation as a linear map}

Let $J \in \R$ be an interval and let $c\in J$,

We define vector space $\D_c$ as,

\[ \D_c = \{ f:J\ra\R : \text{$f$ is differentiable at $c$} \} \]

$\D_c$ is a vector space over $\R$,

Let's define,

\[ \frac{d}{dx} \biggr|_{ x=c } : \D_c \ra \R, \frac{df}{dx} \biggr|_{ x=c } = f'(c) \]

It is trivial to check that, $\frac{d}{dx} \biggr|_{ x=c }$ is a linear map from the vector space $\D_c$ to itself.

\[ \D = \{ f:j \ra \R : \text{$f$ is differential on $J$} \} \] 

\[ \mathfrak{F} = \{ f:J \ra \R \} \]

\[ \frac{d}{dx} : \D \ra \mathfrak{F} , f \ra \frac{df}{dx}  \]

It is again trivial to check that, $\frac{d}{dx}$ is a linear map from the vector space $\D$ to itself.

\begin{notation}[Continous functions]
	Given interval $J$ in $\R$
	\[ \mathcal{C}(J) = \{ f:J \ra \R : \text{$f$ is continous on $J$} \}  \]
\end{notation}

\begin{notation}[Continously differentiable functions]
	Given interval $J$ in $\R$
	\[ \mathcal{C}^1(J) = \{ f:J \ra \R : \text{$f$ is differentiable on $J$ and $f' \in \mathcal{C}(J)$ } \}  \]
\end{notation}

\subsection{Chain Rule}

\begin{theorem}[Chain Rule]
	Given intervals $J$ and $J_1$ in $\R$ and functions $f:J\ra \R$ and $g:J_1 \ra \R$, such that $f(J) \subseteq J_1$. 
	Let $c\in J$, if $f$ is differentiable at $c$ and $g$ is differentiable at $f(c)$ then $(gof)$ is differentiable at $c$ and $(gof)'(c)$ is given by
	\[ (gof)'(c) = g'(f(c))f'(c) \]
\end{theorem}

\begin{proof}
	From theorem 1.7 we have that there exists $f_1:J\ra\R$ such that,
	\[ f(x) = f(c) + f_1(x)  (x-c) \forall x \in J\]
	where $f_1$ is continous at $c$ and $f'(c)=f_1(c)$\\
	From theorem 1.7 we have that there exists $g_1:J_1\ra\R$ such that,
	\[ g(y) = g(f(c)) + g_1(y)  (y-f(c)) \forall y \in J_1 \]
	where $g_1$ is continous at $f(c)$ and $g'(f(c))=g_1(f(c))$\\
	Now,
	\[ (gof)(x) = g(f(x)) \]
	\[ \implies (gof)(x) = g(f(c)) + g_1(f(x)) ( f(x) - f(c) ) \]
	\[ \implies (gof)(x) = g(f(c)) + g_1(f(c)+f_1(x)(x-c))f_1(x)(x-c) \]
	Since $g_1$ and $f_1$ are continous, $g_1(f(c)+f_1(x)(x-c))f_1(x)$ is contious, therefore we can apply theorem 1.7 now and we get, 
	\[ (gof)_1(x) = g_1(f(c)+f_1(x)(x-c))f_1(x) \]
	\[ (gof)'(c) = (gof)_1(c) = g_1(f(c)+f_1(c)(c-c))f_1(c) \]
	\[ (gof)'(c) = g'(f(c))f'(c) \]

\end{proof}

We shall now declare few conjectures to give ourselves oppertunity to look at better examples (just going along with how the course was thought, I the student do not like this),

\begin{conjecture}
	$f(x)=e^x$ is differentiable in $\R$ and $f'(x) = e^x$
\end{conjecture}

\begin{conjecture}
	$f(x)=\text{log}(x)$ is differentiable in $\R$ and $f'(x) = 1/x$
\end{conjecture}

\begin{conjecture}
	$f(x)=\text{sin}(x)$ is differentiable in $\R$ and $f'(x) = \text{cos}(x)$
\end{conjecture}

\begin{conjecture}
	$f(x)=\text{cos}(x)$ is differentiable in $\R$ and $f'(x) = -\text{sin}(x)$
\end{conjecture}

\begin{example}
	(This example uses one of the cojectures)
	\[ f(x) = x^\a, x > 0, \a\in\R \]
	Then,
	\[ f(x) = e^{\a\text{log}(x)} \]
	let, $q(x) = \a \text{log}(x)$ and $p(x) = e^x$, therefore,
	\[ f(x) = (poq)(x) \]
	Now using chain rule we get,
	\[ f'(x) = p'(q(x))q'(x) \]
	\[ \implies f'(x) = e^{\a\text{log}(x)} \frac{\a}{x} \]
	\[ \implies f'(x) = \a x^{\a-1} \]
\end{example}

\subsection{Local minima and maxima}

\begin{definition}[Local Minima]
	Let $f:J\ra\R$ be a given real valued function from an interval in $\R$, we say that $c\in J$ is a local minima of $f$ if $\exists \d \in \R$ such that,
	\begin{enumerate}
		\item \[ (c-\d,c+\d) \subseteq J \]
		\item \[ f(x) \geq f(c) \forall x \in (c-\d,c+\d) \]
	\end{enumerate}
\end{definition}

\begin{definition}[Local Maxima]
	Let $f:J\ra\R$ be a given real valued function from an interval in $\R$, we say that $c\in J$ is a local maxima of $f$ if $\exists \d \in \R$ such that,
	\begin{enumerate}
		\item \[ (c-\d,c+\d) \subseteq J \]
		\item \[ f(x) \leq f(c) \forall x \in (c-\d,c+\d) \]
	\end{enumerate}
\end{definition}

\begin{definition}[Global Minima]
	Let $f:J\ra\R$ be a given real valued function from an interval in $\R$, we say that $c\in J$ is a global minima of $f$ if,
	\[ f(x) \geq f(c) \forall x \in J \]
\end{definition}

\begin{definition}[Global Maxima]
	Let $f:J\ra\R$ be a given real valued function from an interval in $\R$, we say that $c\in J$ is a global maxima of $f$ if,
	\[ f(x) \leq f(c) \forall x \in J \]
\end{definition}

\begin{example}
	$f:[-1,1]\ra\R, f(x) = x^2$, 0 is local and global minima.
\end{example}

\begin{example}
	$f:[0,1]\ra\R, f(x)=x$, 0 is global but not local minuma.
\end{example}

\begin{theorem}
	For $J$ interval in $\R$, let $f:J\ra\R$ and let $c$ be a local minima, if $f$ is differentiable at $c$ then $f'(c)=0$.
\end{theorem}

\begin{proof}
	Since $c$ is a local minima of $f$, there exists $\d >0$ such that,
	\[ (c-\d,c+\d) \subset J \]
	and
	\[ f(c+h) \geq f(c), 0<h<\d \]
	since $f$ is differentiable at $c$ we can write,
	\[ f'(c) =  \lim_{ h \to 0 } \frac{ f(c+h) - f(c) }{ h } \]
	For the above limit, the left hand limit is $\leq 0$ and right hand limit is $\geq 0$ but since $f$ is differentiable, both must be equal and hence must be equal to zero.
\end{proof}

\begin{theorem}
	For $J$ interval in $\R$, let $f:J\ra\R$ and let $c$ be a local maxima, if $f$ is differentiable at $c$ then $f'(c)=0$.
\end{theorem}

\begin{proof}
	The theorem trivially follows fromt the previous theorem.
\end{proof}

We shall see a more generalized form of these theorems later on which includes point of inflection after we prove Taylor's theorem.

\subsection{Rolle's theorem}

\begin{theorem}[Rolle's theorem]
	For $a,b \in \R$, let $f:[a,b] \ra \R$ be such that,
	\begin{enumerate}
		\item $f$ is continous in $[a,b]$.
		\item $f$ is differentiable in $(a,b)$.
		\item $f(a)=f(b)$.
	\end{enumerate}
	Then $\exists c \in (a,b)$ such that, $f'(c) = 0$.
\end{theorem}

\begin{proof}
	Since $f$ is continous and $[a,b]$ is a compact set in $\R$, $f$ will map $[a,b]$ to a compact set in $\R$ and hence $\exists x_1,x_2 \in [a,b]$ such that,
	\[ f(x_1) \leq f(x) \leq f(x_2) \forall x \in [a,b] \]

	\begin{enumerate}
		\item \textit{Case 1:} $f(x_1) = f(x_2) $\\
			$f$ is a constant function and hence $f'(x) = 0 \forall x \in [a,b]$.
		\item \textit{Case 2:} $f(x_1) \leq f(x_2)$\\
			We can claim that, $\{x_1,x_2\} \neq \{a,b\}$ since $f(x_1)\neq f(x_2)$. This now proves that there exists a local minima or local maxima which now completes the proof.
	\end{enumerate}
\end{proof}

\subsection{Mean value theorem}

\begin{theorem}[Mean value theorem]
	For $a,b \in \R$, let $f:[a,b] \ra \R$ be such that,
	\begin{enumerate}
		\item $f$ is continous in $[a,b]$.
		\item $f$ is differentiable in $(a,b)$.
	\end{enumerate}
	Then $\exists c \in (a,b)$ such that,
	\[ f'(c) = \frac{ f(a) - f(b) }{ b-a } \]
\end{theorem}

\begin{proof}
	Define $g:[a,b]\ra\R$,
	\[ g(x) = f(x) - f(a) - \l( \frac{ x-a }{ b-a } \r) ( f(b) - f(a) ) \]
	By algebraic properties of continous real functions, we know $g$ is continous in $[a,b]$ and by algebraic properties of differentiation, we also know that $g$ is differentiable in $(a,b)$. Now by applying Rolle's theorem to $g$, $\exists c \in (a,b)$ such that
	\[ g'(c) = 0 \]
	\[ \implies  f'(c) - \frac{ f(b)-f(a) }{ b-a } = 0 \]
	\[ \implies f'(c) = \frac{ f(b)-f(a) }{ b-a } \]
\end{proof}

\begin{example}[costant function]
	Let $J$ be an interval in $\R$ and let $f:J\ra\R$ be differentiable on $J$ and assume $f'(x)=0 \forall x \in J$, then $f$ is a constant function.
\end{example}

\begin{proof}
	for $x,y\in J$ by mean value theorem, $f(x)-f(y)=f'(c)(x-y)$, for some $c\in J$ and since $f'=0$, $f(x) = f(y) \forall x,y \in J$.
\end{proof}

\begin{example}[non decreasing function]
	Let $J$ be an interval in $\R$ and let $f:J\ra\R$ be differentiable on $J$ and assume $f'(x)>0 \forall x \in J$, then $f$ is a non decreasing function.
\end{example}

\begin{proof}
	for $x,y\in J$ by mean value theorem, $f(x)-f(y)=f'(c)(x-y)$, for some $c\in J$ and since $f'>0$, $f(x) \geq f(y) \forall x,y \in J$ with $x\geq y$.
\end{proof}

\begin{definition}[Lipschitz functions]
	For $J$ interval in $\R$, let $f:J\ra\R$. $f$ is said to be lipshcitz if $\exists M>0$ such that, $|f(x)-f(y)|\leq M|x-y|$.
\end{definition}

By definition, it is clear that lipshitz functions are uniformly continous.

\begin{theorem}
	For $J$ interval in $\R$, let $f:J\ra\R$, if 
	\begin{enumerate}
		\item $f:J\ra\R$ is differentiable.
		\item $f':J\ra\R$ is bounded.
	\end{enumerate}
	Then $f$ is lipschitz.
\end{theorem}

\begin{proof}
	Let $x,y\in J$, assume $x<y$ now applying mean value theorem on the closed interval $[x,y]$ we get that $\exists c \in (x,y)$ such that,
	\[ f(x) - f(y) = f'(c)(x-y) \]
	and now since $f'$ is bounded, there exists $M>0$ such that,
	\[ |f'(c)| \leq M \forall c \in J \]
	From this we have,
	\[ f(x) - f(y) \leq M|x-y| \forall x,y \in J \]
\end{proof}

\begin{example}
	$f(x) = \sqrt{x}, x \in [0,1]$, $f'(x) = x^{-1/2}/2$, $f'$ is bounded for the given domain and hence is lipschitz.
\end{example}

\begin{definition}[Continously differentiable]
	For $J$ interval in $\R$, let $f:J\ra\R$ be differentiable function, $f$ is said to be continously differentiable if $f':J \ra \R$ is a continous function.
\end{definition}

\subsection{Inverse function theorem}

\begin{theorem}[Inverse function theorem]
	For $J$ interval in $\R$, let $f:J\ra\R$, if $f$ is continously differentiable on $J$ and assuming $f'(x) \neq 0 \forall x \in J$ then the following hold
	\begin{enumerate}
		\item $f$ is strictly monotone.
		\item $f(J)$ is an interval.
		\item $f$ has an inverse $g$.
		\item $g$ is differentiable.
	\end{enumerate}
\end{theorem}

\begin{proof}

	$f$ is differentiable in $J$ and hence is continous on $J$. Continous images of connected sets are connected and, in $\R$ all connected sets are intervals and all intervals are connected sets hence concludes the proof of II).\\

	If f is not strictly monotone then there exists $x,y,z \in J$ such that $x<y<z$ and, 
	
	\[ f(y) \leq \text{min} \{f(x),f(z)\} \text{ or } f(y) \geq \text{max} \{f(x),f(z)\}\]
	
	With out loss of generality assume, 
	
	\[ f(y) \leq \text{min} \{f(x),f(z)\} \]

	Then either,
	
	\[ [ f(y),f(x) ] \subseteq  [ f(y),f(z) ] \]
	
	or
	
	\[ [ f(y),f(z) ] \subseteq  [ f(y),f(x) ] \]
	
	Without loss of generality assume, 
	
	\[ [ f(y),f(x) ] \subseteq  [ f(y),f(z) ] \]

	Since $f(x) \in [ f(y),f(z) ]$ by intermediate value theorem we are able to tell that,
	
	\[ \exists z_1 \in [y,z] \text{ such that } f(z_1) = f(x) \]
	
	and now by mean value theorem we are able to tell that,
	
	\[ \exists c \in [x,z_1] \text{ such that,}\]
	
	\[ f'(c) = \frac{ f(x) - f(z_1) }{ x-z_1 } = 0 \]

	But this leads to a contradiction that $f$ is such that, $f'(i) \neq 0 \forall i \in J$ and hence our assumption that $f$ is not strictly monotone is wrong.\\

	If $a,b \in J$ such that $a \neq b$ then without loss of generality assume $a<b$ then $f(a) < f(b)$ or $f(b) < f(a)$ and hence $f(a) \neq f(b)$ if $a \neq b$ which makes $f$ an injective mapping and hence,
	\[ g: f(J) \ra J, f(x) \ra x \]
	is a valid function and hence inverse of $f$ exists.\\
	
	Let $c \in f(j)$, we need to show that
	
	\[ \lim_{y \ra c } \frac{ g(y) - g(c) }{ y-c } \]
	
	exists to show that $g$ is differentiable.\\

	Let $y_n \ra c$ with $y_n \in f(J)$ therefore there exists sequence $\{x_n\}$ in $J$ such that,

	\[ f(x_n) = y_n \]

	and there exists $d \in J$ such that,

	\[ f(d) = c \]

	Now,

	\[ \frac{ g(y_n) - g(c) }{ y_n - c } = \frac{ g(f(x_n)) - g(f(d)) }{ f(x_n) - f(d) } = \frac{ x_n - d }{ f(x_n) - f(d) } = \frac{1}{\frac{ f(x_n) - f(d) }{ x_n - d }} \]
	
	\[ \implies \lim_{n \ra \infty} \frac{ g(y_n) - g(c) }{ y_n - c } = \lim_{n \ra \infty} \frac{1}{\frac{ f(x_n) - f(d) }{ x_n - d }} \]
	
	\[ \implies g'(c) = 1/f'(d) = 1/f'(g(c)) \]

	This proves that $g$ the inverse of $f$ is differentiable.

\end{proof}

\subsection{ Cauchy's mean value theorem }

\begin{theorem}[Cauchy's mean value theorem]

	For $a,b \in \R$ let $f,g: [a,b] \ra \R$ be functions both differentiable on $(a,b)$ and if $g(x) \neq 0 \forall x\in [a,b]$ then,
	\begin{enumerate}
		\item $g(b) \neq g(a)$
		\item $\exists c \in (a,b)$ such that
			\[ \frac{ f(b) - f(a) }{ g(b) - g(a) } = \frac{ f'(c) }{ g'(c) } \]
	\end{enumerate}
\end{theorem}

\begin{proof}

	The proof of I) follows from previous theorem now let,

	\[ G(x) = (f(b)-f(a))g(x) - (g(b)-g(a))f(x) \]
	
	Now we have,

	\[ G(a) = (f(b)-f(a))g(a) - (g(b)-g(a))f(a) \]
	
	\[ \implies G(a) = f(b)g(a) - g(b)f(a) \]

	\[ G(b) = (f(b)-f(a))g(b) - (g(b)-g(a))f(b) \]
	
	\[ \implies G(b) = f(b)g(a) - g(b)f(a) \]

	Since $G(a) = G(b)$ we are able to use Rolle's theorem for $f$ hence $\exists c \in (a,b)$ such that,

	\[ G'(c) = 0 \]
	
	\[ \implies G'(c) = (f(b)-f(a))g'(c) - (g(b)-g(a))f'(c) \]
	
	\[ \implies (f(b)-f(a))g'(c) - (g(b)-g(a))f'(c) = 0 \]

	\[ \implies (f(b)-f(a))g'(c) = (g(b)-g(a))f'(c) \]

	\[ \implies \frac{ f(b) - f(a) }{ g(b) - g(a) } = \frac{ f'(c) }{ g'(c) } \]

\end{proof}

\begin{theorem}[Darboux theorem]
	Given $a,b \in \R$ let $f:[a,b] \ra \R$ be a differential function and assume that $f'(a) \leq k \leq f'(b)$ then $\exists c \in [a,b]$ such that $f'(c) = k$.
\end{theorem}

\begin{proof}

	Let, 

	\[ F:[a,b] \ra \R, x \ra f(x) - kx \]

	$F$ is differentiable and hence a continous function and since continous mapping of a compact connected sets are compact and connected, the image of $F$ has a minimum and a maximum.\\

	Since $F$ is differentiable at $a$ we have that given $\e > 0$ $\exists$ $\d >0$ such that,

	\[ \l| \frac{ F(x) - F(a) }{ x - a } - F'(a) \r| < \e \text{, } \forall \text{ } x \in (a,a+\d) \]

	\[ \implies F'(a) - \e < \frac{ F(x) - F(a) }{ x - a } < F'(a) + \e \text{, } \forall \text{ } x \in (a,a+\d) \]

	Since $F'(a) < 0$ we can let $\e = -F'(a)/2$, then there exists $\d_1$ such that

	\[ \frac{ F(x) - F(a) }{ x - a } < \frac{ F'(a) }{ 2 } \text{, } \forall \text{ } x \in (a,a+\d_1) \]

	\[ \implies F(x) - F(a) > x - a > 0 \text{, } \forall \text{ } x \in (a,a+\d_1) \]

	Therefore $a$ is not the point of global maximum for $F$.\\
	
	Since $F$ is differentiable at $b$ we have that given $\e > 0$ $\exists$ $\d >0$ such that,

	\[ \l| \frac{ F(x) - F(b) }{ x - b } - F'(b) \r| < \e \text{, } \forall \text{ } x \in (b-\d,b) \]

	\[ \implies F'(b) - \e < \frac{ F(x) - F(b) }{ x - b } < F'(b) + \e \text{, } \forall \text{ } x \in (b-\d,b) \]

	Since $F'(b) > 0$ we can let $\e = F'(b)/2$, then there exists $\d_2$ such that

	\[ \frac{ F'(b) }{ 2 } < \frac{ F(x) - F(b) }{ x - b } \text{, } \forall \text{ } x \in (b-\d,b) \]

	\[ \implies F(x) - F(a) > x - a > 0 \text{, } \forall \text{ } x \in (a,a+\d_1) \]

	Therefore $a$ is not the point of global maximum for $F$.\\
	
	Then there must exist a maximum of $F$ in the interior of $F([a,b])$ making it also a local maximum, therefore there exists a point $c \in [a,b]$ such that,

	\[ F'(c) = 0 \]

	\[ \implies f'(c) = k \]

\end{proof}

\subsection{L'Hospital rule}

\begin{theorem}[L'Hospital rule]
	Suppose $f$ and $g$ are real and differentiable in $(a,b)$ and $g'(x) \neq 0$ for all $x \in (a,b)$ for $a,b \in \R$. Suppose,
	\[ \frac{ f'(x) }{ g'(x) } \ra A \text{ as } x \ra a \]
	If, 
	\[ f(x) \ra 0 \text{ and } g(x) \ra 0 \text{ as } x \ra a \]
	or if,
	\[ g(x) \ra \infty \text{ as } x \ra a \]
	then,
	\[ \frac{ f(x) }{ g(x) } \ra A \text{ as } x \ra a \]
\end{theorem}

\begin{proof}

	\textit{ Case 1: $A \neq \pm \infty$} \\
	Choose a real number $q > A$ and choose another real number $r$ such that $A < q < r$,\\
	because $\frac{ f'(x) }{ g'(x) } \ra A \text{ as } x \ra a$, $\exists$ $c \in (a,b)$ such that $a < x < c$ and 
	\[ \frac{ f'(x) }{ g'(x } < r. \]
	
	If $a<x<y<c$ then by cauchy's mean value theorem\\

	(to be continued)
	
\end{proof}

\end{document}
